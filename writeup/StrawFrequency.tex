
%----------------------------------------------------------------------------------------
%	PACKAGES AND OTHER DOCUMENT CONFIGURATIONS
%----------------------------------------------------------------------------------------




\documentclass[twoside]{article} 
\usepackage{textcomp}

\usepackage{lipsum} % Package to generate dummy text throughout this template

\usepackage[sc]{mathpazo} % Use the Palatino font
\usepackage[T1]{fontenc} % Use 8-bit encoding that has 256 glyphs
\linespread{1.05} % Line spacing - Palatino needs more space between lines
\usepackage{microtype} % Slightly tweak font spacing for aesthetics

\usepackage{amsmath}

\usepackage[hmarginratio=1:1,top=32mm,columnsep=20pt]{geometry} % Document margins
\usepackage{multicol} % Used for the two-column layout of the document
\usepackage[hang, small,labelfont=bf,up,textfont=it,up]{caption} % Custom captions under/above floats in tables or figures
\usepackage{booktabs} % Horizontal rules in tables
\usepackage{float} % Required for tables and figures in the multi-column environment - they need to be placed in specific locations with the [H] (e.g. \begin{table}[H])
\usepackage{hyperref} % For hyperlinks in the PDF

\usepackage{lettrine} % The lettrine is the first enlarged letter at the beginning of the text
\usepackage{paralist} % Used for the compactitem environment which makes bullet points with less space between them

\usepackage{abstract} % Allows abstract customization
\renewcommand{\abstractnamefont}{\normalfont\bfseries} % Set the "Abstract" text to bold
\renewcommand{\abstracttextfont}{\normalfont\small\itshape} % Set the abstract itself to small italic text

\usepackage{titlesec} % Allows customization of titles
\renewcommand\thesection{\Roman{section}} % Roman numerals for the sections
\renewcommand\thesubsection{\Roman{subsection}} % Roman numerals for subsections
\titleformat{\section}[block]{\large\scshape\centering}{\thesection.}{1em}{} % Change the look of the section titles
\titleformat{\subsection}[block]{\large}{\thesubsection.}{1em}{} % Change the look of the section titles

\usepackage{fancyhdr} % Headers and footers
\pagestyle{fancy} % All pages have headers and footers
\fancyhead{} % Blank out the default header
\fancyfoot{} % Blank out the default footer
\fancyhead[C]{Straw Frequencies $\bullet$ September 2019 $\bullet$ Mu2e-doc-6392-v5 } % Custom header text
\fancyfoot[RO,LE]{\thepage} % Custom footer text

\usepackage[english]{babel}
\usepackage{graphicx}

%change the margins
\addtolength{\oddsidemargin}{-.875in}
\addtolength{\evensidemargin}{-.875in}
\addtolength{\textwidth}{1.75in}



%----------------------------------------------------------------------------------------
%	TITLE SECTION
%---------------------------------------------------------------------------------------- 

\title{\vspace{-15mm}\fontsize{24pt}{10pt}\selectfont\textbf{Natural Frequencies of Straw Tubes}} % Article title

\author{
\large
\textsc{Jason Bono}\\[2mm]
\normalsize Fermilab \\ 
\normalsize \href{mailto:jason.s.bono@gmail.com}{jason.s.bono@gmail.com} 
\vspace{-5mm}
}
\date{}

%----------------------------------------------------------------------------------------

\begin{document}

\maketitle % Insert title

\thispagestyle{fancy} % All pages have headers and footers

%----------------------------------------------------------------------------------------
%	ABSTRACT
%----------------------------------------------------------------------------------------

\begin{abstract}

\noindent For straw-tubes, the relationship between their tension and natural frequencies is needed for most tension measurement techniques.  Furthermore, recent data are in disagreement with previous treatments.  In this document, the straw-tube is treated as a medium with two restoring forces; one arising from the tension applied to the straws, and the other arising from their flexural rigidity, or bending stiffness. The resultant equation of motion with the appropriate boundary conditions gives rise to a functional form for natural frequencies that matches the recent measurements.  However, the model's parameter values as calculated by theory and past experiment are in disagreement. This disagreement is then resolved by a new, larger experiment that reproduces the theory model and its parameter values under a wide variety of experimental conditions.


\end{abstract}

%----------------------------------------------------------------------------------------
%	ARTICLE CONTENTS
%----------------------------------------------------------------------------------------

\begin{multicols}{2} % Two-column layout throughout the main article text



%%%%%%%%%%%%%%%%%%%
%%%%%%%%%%%%%%%%%%%
%%%%%%%%%%%%%%%%%%%
\section{Restoring Forces}
Here we consider the straw-tube as a medium with two restoring forces: one arising from the tension applied to the straws, and the other from their \emph{flexural rigidity}, or \emph{bending stiffness}.  After the initial displacement, the straw motion is driven by these the two forces alone. Thus for a straw segment of width $dx$, the equation of motion is
\begin{equation}
\mu dx \frac{\partial^2y}{\partial t^2}  = \frac{\partial F_t}{\partial x}dx + \frac{\partial F_b}{\partial x}dx 
\end{equation}
where $F_t$ denotes the force that arises from the tension, and similarly, $F_b$ from the bending stiffness. More specifically, $F_t$ is the vertical component of tension on the boundary of the segment, which has a magnitude given by
\begin{equation}
F_t = T \sin\theta 
\end{equation}
where $T$ denotes the applied tension. Since for small angles, $\sin\theta \approx \tan\theta \approx \frac{\partial y}{\partial x}$, so we get
\begin{equation}
\mu \frac{\partial ^2y}{\partial t^2}  =  T\frac{\partial^2 y}{\partial x^2} + \frac{\partial F_b}{\partial x} 
\end{equation}


Next, we evaluate $F_b$ which can be written in terms of a \emph{bending moment} over the length of a straw segment,
\begin{equation}
F_b  = -\frac{\partial M}{\partial x} 
\end{equation}
The bending stiffness, $B$, is defined as the bending moment divided by the curvature, $\kappa$, that it induces on an otherwise straight segment
\begin{equation}
 B \equiv -\frac{M}{\kappa}
\end{equation}
For small deflections, the curvature can be approximated as
\begin{equation}
\kappa \approx \frac{\partial^2y}{\partial x^2}
\end{equation}
so that the bending moment can be written as
\begin{equation}
M = - B \frac{\partial^2y}{\partial x^2}
\end{equation}
and therefore
\begin{equation}
	\frac{\partial F_b}{\partial x} =  -B \frac{\partial^4y}{\partial x^4}
\end{equation}
Plugging the above expression back into the equation of motion reveals a fourth-order differential equation
\begin{equation}
\boxed{
\mu \frac{\partial^2y}{\partial t^2} = T \frac{\partial^2y}{\partial x^2} - B \frac{\partial^4y}{\partial x^4}
\label{eq:it}
}
\end{equation}

Incidentally, the bending stiffness can be evaluated from the \emph{young's modulus}, $E$ of the material and the \emph{second moment of area} of its cross section, $I$,
\begin{equation}
B = E I
\end{equation}
Note that throughout this document, the cross section refers to that which has its normal vector parallel to the straw length.


%%%%%%%%%%%%%%%%%%%
%%%%%%%%%%%%%%%%%%%
%%%%%%%%%%%%%%%%%%%
\section{General Solution to the Equation of Motion}
Beginning with Equation~\ref{eq:it}, and making the $x$ and $t$ dependence of the vertical displacement explicit, we have 
\begin{equation}
\mu \frac{\partial^2y(x,t)}{\partial t^2} = T \frac{\partial^2y(x,t)}{\partial x^2} - B \frac{\partial^4y(x,t)}{\partial x^4}
\end{equation}
Introducing the quantity
\begin{equation}
\eta  \equiv \frac{B}{T}
\end{equation}
which has dimensions of area, and the group velocity for a wave on a string
\begin{equation}
c  \equiv \sqrt{\frac{T}{\mu}}
\end{equation}
we may write
\begin{equation}
\frac{1}{c^2} \frac{\partial^2y(x,t)}{\partial t^2} = \frac{\partial^2y(x,t)}{\partial x^2} - \eta \frac{\partial^4y(x,t)}{\partial x^4}
\label{eq:mot}
\end{equation}
It will also be useful later to define the nominal string frequencies
\begin{equation}
	\mathcal{F}_n = \frac{nc}{2L}
\end{equation}
where $n$ is an integer and $L$ is the length of the straw, and finally the dimensionless quantity
\begin{equation}
\xi  \equiv \frac{\sqrt{\eta}}{L}
\label{eq:xi}
\end{equation}

It is advantageous to solve Equation~\ref{eq:mot} in Fourier space. The Fourier transform of the solution $y$ with respect to time is
\begin{equation}
	F(x,\omega) = \int_{-\infty}^{\infty} y(x,t)\exp(-i \omega t)dt
\end{equation}
which yields the separable equation
\begin{equation}
\frac{- \omega^2}{c^2}  Y(x,\omega) = \frac{d^2Y(x,\omega)}{dx^2} - \eta \frac{d^4Y(x,\omega)}{dx^4}
	\label{eq:for}
\end{equation}
 We look for solutions of the form
\begin{equation}
	Y(x,\omega) = X(x)W(\omega)
\end{equation}
which, when inserted into Equation~\ref{eq:for}, gives
\begin{equation}
\frac{- \omega^2}{c^2}  X = X^{\prime\prime} - \eta X^{\prime\prime \prime \prime}
\end{equation}
The above has a characteristic equation
\begin{equation}
0 = \frac{ \omega^2}{c^2}   + \lambda^2 - \eta \lambda^4
\label{eq:car}
\end{equation}
provided  a solution of the form
\begin{equation}
	X(x) = C\exp(\lambda x)
\end{equation}
Equation~\ref{eq:car} is reducible to a second order polynomial in $\lambda$ with solutions 
\begin{equation}
	\begin{cases}
	\lambda^\pm_1 = \pm  \sqrt{\frac{1 - \sqrt{1 + \frac{4\omega^2 \eta }{c^2}  }  }{2\eta }} \equiv \pm ik_1 \\ 
	\lambda^\pm_2 = \pm  \sqrt{\frac{1 + \sqrt{1 + \frac{4\omega^2 \eta }{c^2}  }  }{2\eta }} \equiv \pm k_2
	\end{cases} 
		\label{eq:lambdas}
\end{equation}
where we have defined $k_1$ and $k_2$ so that they are necessarily real. Since $\lambda_1$ and $\lambda_2$ are respectively imaginary and real, the general solution consists of both an oscillating and an exponential part 
\begin{equation}
\begin{cases}
Y_1(x,\omega) = c^+_1 \exp(ik_1 x) + c^-_1 \exp(-ik_1 x)  \\ 
Y_2(x,\omega) = c^+_2 \exp(k_2 x) + c^-_2 \exp(-k_2 x) \\
\end{cases} 
\label{eq:gensol}
\end{equation}
with, by virtue of the superposition principle, 
\begin{equation}
	Y(x,\omega) = Y_1(x,\omega) +  Y_2(x,\omega) 
\end{equation}

The solution to the equation of motion, under appropriate boundary conditions, will reveal the permissible values of $k_1$ and $k_2$. Equation~\ref{eq:lambdas} can then be used to obtain the natural frequencies since it can be inverted as
\begin{equation}
\omega(k_1) =  c k_1 \sqrt{ k^2_1 \eta + 1}
\label{eq:omega}
\end{equation}






%%%%%%%%%%%%%%%%%%%
%%%%%%%%%%%%%%%%%%%
%%%%%%%%%%%%%%%%%%%
\section{Perfectly Glued Straws}
The straws are glued at both ends so that they may not oscillate or rotate.  The boundary conditions are thus
\begin{equation}
\label{eq:clamped}
\begin{cases}
Y(x,\omega)|^{-L/2} = Y(x,\omega)|^{L/2} = 0 \\
\frac{dY(x,\omega)}{dx}|^{-L/2} = \frac{dY(x,\omega)}{dx}|^{L/2} = 0 
\end{cases} 
\end{equation}
Imposing the above conditions on Equation~\ref{eq:gensol}, we eventually end up with two transcendental equations
\begin{equation}
\begin{cases}
k_1 \tan(k_1 \frac{L}{2}) = - k_2 \tanh(k_2 \frac{L}{2}) \\
k_1 \tanh(k_2 \frac{L}{2}) =  k_2 \tan(k_1 \frac{L}{2})  
\end{cases} 
\label{eq:trans}
\end{equation}

We may numerically solve the above equations for the oscillatory term, $K_1$, by eliminating $K_2$ using
\begin{equation}
k^2_1 + k^2_2 = \frac{1}{\eta}
\label{eq:rel}
\end{equation}
which follows from Equation~\ref{eq:lambdas}. The values of $k_1$ can then be inserted into Equation~\ref{eq:omega} to obtain the natural frequencies.  This was carried out by Harvey Fletcher~\cite{fletcher} who approximated 
 \begin{equation}
 	\omega_n \approx (n \pi \frac{c}{L})(1 + 2\xi + 3 \xi^2)\sqrt{1 + n^2\pi^2\xi^2}
 \label{eq:full}
 \end{equation}
 where $\xi$ is defined in Equation~\ref{eq:xi}.
 Taking up to the second order terms in the Taylor expansion about $\xi = 0$ in Equation~\ref{eq:full}, we get
  \begin{equation}
  \omega_n \approx (n \pi \frac{c}{L}) (1 + 2\xi + 4\xi^2 + \frac{n^2\pi^2}{2}\xi^2)
  \end{equation}
 hence
  \begin{equation}
  	f_n \approx \mathcal{F}_n (1 + 2\xi + 4\xi^2 + \frac{n^2\pi^2}{2}\xi^2)
 \label{eq:hence}
 \end{equation}
  
For straws that are long, or under a large tension, the second order terms in $\xi$ may also be neglected. Thus
 \begin{equation}
 \boxed{
 	f_n \approx \mathcal{F}_n (1 + 2\xi )
 }
 \end{equation}
 or, more explicitly 
  \begin{equation}
  f_n \approx  \frac{n}{2L}\sqrt{\frac{T}{\mu}}  (1 + \frac{2}{L}  \sqrt{\frac{B}{T}} )
 \label{eq:end}
  \end{equation}






%%%%%%%%%%%%%%%%%%%
%%%%%%%%%%%%%%%%%%%
%%%%%%%%%%%%%%%%%%%
\section{Predicted v.s. Experimental Form}
We see from Equation~\ref{eq:hence} that to first order in $\xi$, the bending stiffness introduces one extra term in what would otherwise be the natural frequencies of a simple string. This term increases the natural frequency by an amount
\begin{equation}
\Delta f_n =  2 \xi \mathcal{F}_n
\end{equation}
Defining the constant 
\begin{equation}
\mathcal{C}_n \equiv n \sqrt{\frac{B}{\mu}}
\label{eq:c}
\end{equation}
we can write
\begin{equation}
\Delta f_n =  \frac{\mathcal{C}_n}{L^2}
\end{equation}


It was found experimentally by \cite{david} that the first natural frequency of the Mylar straw-tubes used in Mu2e followed 
\begin{equation}
\boxed{
	f^\text{exp}_1 = \frac{\mathcal{K}^\text{exp}}{2L}\sqrt{m} + \frac{\mathcal{C}^\text{exp}}{L^2}
}
\end{equation}
where $m$ is the mass of the weight used for tensioning the straw.  The above expression is equivalent to Equation~\ref{eq:end} for $n=1$, provided that
\begin{equation}
	\mathcal{C}^\text{exp} = \mathcal{C}_1
\end{equation}
where $\mathcal{C}_1$ is defined in Equation~\ref{eq:c},
and 
\begin{equation}
\mathcal{K}^\text{exp} = \sqrt{\frac{g}{\mu}} \equiv \mathcal{K} 
\end{equation}
where $g$ is the acceleration due to gravity.





\section{Predicted v.s. Experimental Parameter Values}
\label{sec:prop}
Here we compute the values of $\mathcal{K}$ and $\mathcal{C}_n$ based off the properties of Mylar and compare with what was found experimentally. 
Neglecting the metallization layer, the Mu2e straws are Mylar tubes, 5 mm in diameter and with a wall thickness of 18 $\mu$m.  In terms of mass however, the effective wall thickness is 15 $\mu$m.

For an annulus with inner and outer diameter $r_i = 2.485 \text{ mm}$ and $r_o = 2.5 \text{ mm}$, we have  a cross sectional area of
\begin{equation}
A =  \text{2.35e-7 m}^2
\end{equation}
The second moment of area, assuming 18 $\mu$m wall thickness is
\begin{equation}
I = \frac{\pi}{4}(r^4_o - r^4_1) = \text{8.931e-13 m}^4
\end{equation}
When assuming 15 $\mu$m wall thickness one gets
\begin{equation}
I = \frac{\pi}{4}(r^4_o - r^4_1) = \text{7.297e-13 m}^4
\end{equation}




The Young's modulus of Mylar is 
\begin{equation}
	E = \text{4.90e9 Pa}
\end{equation}
which gives a bending stiffness (assuming 18)
\begin{equation}
B = EI = \text{  4.377e-3  N$\cdot$m}^2
\label{eq:b}
\end{equation}
which gives a bending stiffness (assuming 15)
\begin{equation}
B = EI = \text{  3.575e-3  N$\cdot$m}^2
\end{equation}
When assuming 12 $\mu$m wall thickness one gets
\begin{equation}
B = 0.00286562303
\end{equation}


The density of Mylar is
\begin{equation}
\rho = \text{1390 kg m}^{-3}
\end{equation}
Thus the linear density is
\begin{equation}
\mu = A \rho = \text{   3.266e-4  kg m}^{-1}
\end{equation}
The constant $\mathcal{K}$ is thus
\begin{equation}
\boxed{
	\mathcal{K} = \sqrt{\frac{g}{\mu}} = \text{ 173.2   m s}^{-1} \text{kg}^{-1/2}
}
\end{equation}  
as compared with what was found experimentally, 
\begin{equation}
\mathcal{K}^\text{exp}  = \text{(144.93 $\pm$ 2.49) m s}^{-1} \text{kg}^{-1/2}
\end{equation}
Finally, the constant $\mathcal{C}_n$ is
\begin{equation}
\boxed{
\mathcal{C}_n =  n \sqrt{\frac{B}{\mu}} = n\cdot\text{ 3.66  m}^{2} \text{s}^{-1}
}
\end{equation} 
as compared with what was found experimentally,
\begin{equation}
	\mathcal{C}^\text{exp}_1 = \text{ (6.18 $\pm$ 0.34)  m}^{2} \text{s}^{-1}
\end{equation} 

%One source of uncertainty in the predicted values of $\mathcal{K}$ and $\mathcal{C}$ come from neglecting the glue and metallization layers on the Mylar.  The exclusion of this material results in an underestimation of both the linear mass density and the bending stiffness. This should result in a overestimate of $\mathcal{K}$ and an underestimate for $\mathcal{C}_n$, as is what seems to be the case.  Sources of experimental uncertainty are discussed in \cite{david}.  

Next we evaluate the effect of the metallization layers for the Lamina straws used by \cite{david}. The second moment of area of a 500 \r{A} layer of copper on the inside of the tube is
\begin{equation}
	I_c = \frac{\pi}{4}(0.0025^4 - 2.49995^4) = 2.4540\text{e-15 m}^4 
\end{equation}
while for 1000 \r{A} of aluminum on the outside it is
\begin{equation}
I_a = \frac{\pi}{4}( 0.0025001^4 - 0.0025^4) = 4.9090\text{e-15 m}^4 
\end{equation}

The Young's modulus of copper is
\begin{equation}
E_c = 1\text{1.17e11 Pa} 
\end{equation}
and for aluminum, it is
\begin{equation}
E_a = \text{7e10 Pa} 
\end{equation}
However, it has been found that the aluminum layer on similar straws oxidizes all the way through, forming a layer of alumina which has a Young's modulus around
\begin{equation}
E_a \approx \text{4e11 Pa} 
\end{equation}
The copper layer also oxidizes, but the Young's modulus of CuO is not greatly different than that of Cu.
So, the bending stiffness of the copper and aluminum layers are respectively
\begin{equation}
B_c = \text{2.87e-4  N$\cdot$m}^2
\end{equation}
and
\begin{equation}
B_a = \text{1.9636e-3  N$\cdot$m}^2
\end{equation}
and the total bending stiffness is given by adding the layers
\begin{equation}
B^\text{lamina} = B + B_a + B_c = \text{6.62709e-3  N$\cdot$m}^2
\label{eq:totb}
\end{equation}
This gives a modified prediction for the parameter
\begin{equation}
\boxed{
	\mathcal{C}^\text{lamina}_n =  n\cdot\text{ 4.50 m}^{2} \text{s}^{-1}
}
\end{equation} 
which is also smaller than the experimental value.



\section{Virtual Mass}
The density of air is 1.225 kg/m$^3$, the cross sectional area is 0.000019625 m$^-2$ thus the added mass from air inside the straw is 
\begin{equation}
\mu_{\text{air}} = \text{2.404e-5  kg/m}
\end{equation}
and the virtual mass adds the same amount,
\begin{equation}
\mu_{\text{virtual}} = \text{2.404e-5  kg/m}
\end{equation}
thus the total linear mass density is 
\begin{equation}
\mu_{\text{total}} = \text{3.7468e-4  kg/m}
\end{equation}
Alternatively, adding the two air terms to the measured linear density gives
\begin{equation}
\mu_{\text{total}} = \text{3.9608e-4  kg/m}
\end{equation}
which gives 
\begin{equation}
\boxed{
	\mathcal{K} = \sqrt{\frac{g}{\mu}} = \text{ 157.29  m s}^{-1} \text{kg}^{-1/2}
}
\end{equation}  


%What remains is to justify neglecting second order and higher terms in the dimensionless parameter $\xi^2$. The straws are initially tensioned at 0.7 kgf, or 6.8 n.  To be conservative we suppose the tension in a straw is 2 n.  We then have
%\begin{equation}
%	\xi \approx \frac{0.06}{L} 
%\end{equation}
%where $L$ is measured in meters.
%To illustrate the last point, let us assume that the excluded material is responsible for the discrepancy in $\mathcal{K}$, then the true linear mass density is 
%\begin{equation}
%	\mu^\text{exp} = \mu (\frac{\mathcal{K}^\text{exp}}{\mathcal{K}})^2
%\end{equation}




%%%%%%%%%%%%%%%%%%%
%%%%%%%%%%%%%%%%%%%
%%%%%%%%%%%%%%%%%%%
\section{Boundary Effects}
\label{sec:boun}
The angled straw-insert introduces uncertainty on the location and quality of the first point constraint.  This could effect the natural frequencies in two ways: (1), it creates uncertainty in the straw length, and (2), it partially lifts the boundary constraints.

Neglecting second order terms in $\delta L$, Equation~\ref{eq:end} us that the relative uncertainty in natural frequency is
\begin{equation}
\frac{\delta f_n}{f_n} =|\frac{\partial f_n}{\partial L}| \frac{\delta L}{f_n} \approx |\frac{\partial \mathcal{F}_n}{\partial L}|\frac{\delta L}{\mathcal{F}_n} = \frac{\delta L}{L}
\end{equation} 
Which is at most a 2.5 \% effect based on the geometry of the straw inserts.

As mentioned earlier, the straws are glued at both ends so that they may not oscillate. However, if the glue does not cover the straw azimuthally, but rather constrains the straw at a single point, the cross section could undergo small rotations.  This means that the first derivative of the solution $Y^\prime(x,\omega) \neq 0$ at the endpoints. In a frictionless scenario, the torque, and hence $Y^{\prime\prime}(x,\omega)$  is zero at the endpoints. Formally,
\begin{equation}
\label{eq:pinned}
\begin{cases}
Y(x,\omega)|^{-L/2} = Y(x,\omega)|^{L/2} = 0 \\
\frac{d^2Y(x,\omega)}{dx^2}|^{-L/2} = \frac{d^2Y(x,\omega)}{dx^2}|^{L/2} = 0 
\end{cases} 
\end{equation}
Imposing the above conditions on Equation~\ref{eq:gensol} leads to
\begin{equation}
\sin(k_1 L)\sinh(k_2L) = 0
\label{eq:sin}
\end{equation}
and 
\begin{equation}
k^2_1 + k^2_2 = 0
\end{equation}
whose only solution is the trivial one since $k_1$ and $k_2$ are real. The solutions to Equation~\ref{eq:sin} are
\begin{equation}
k_1 = \frac{n\pi}{L}
\end{equation}
where n is an integer. Plugging the allowed values of $k_1$ into Equation~\ref{eq:omega} gives
\begin{equation}
\omega_n = n \pi \frac{c}{L} \sqrt{ \frac{n^2\pi^2\eta}{L^2} + 1}
\end{equation} 
hence
\begin{equation}
\boxed{
	f_n =  \mathcal{F}_n \sqrt{n^2\pi^2\xi^2 + 1}
}
\label{eq:pinsol}
\end{equation}
or, more explicitly
\begin{equation}
\label{eq:clamped2}
f_n =  \frac{n}{2L}\sqrt{\frac{T}{\mu}}  \sqrt{\frac{n^2\pi^2B}{TL^2} + 1}
\end{equation}
Taking up to the second order terms in the Taylor expansion about $\xi = 0$ in Equation~\ref{eq:pinsol} gives
\begin{equation}
f_n \approx \mathcal{F}_n (1 + \frac{n^2\pi^2\xi^2}{2})
\label{eq:pinapprox}
\end{equation}
which with the appropriate substitutions is what Feynman found for a piano wire~\cite{dick}. 
Incomplete gluing or constraining of the straw thus mitigates the effect that bending stiffness has on its natural frequencies, reducing the frequencies closer to that of a string under tension. 


%It is important to note that other effects, such as the straw-out-of-straightness, stretching, and further contributions to the uncertainty in straw length have been ignored. The first two effects have been considered elsewhere. The latter mentioned effect however has not been evaluated since it has only become an issue with the most recent gluing procedure. In particular, the true straw length depends upon the amount by which the glue creeps up the straw length, within the insert holes, which is itself, dependent on the amount of glue used and the location of the injection site . 


%%%%%%%%%%%%%%%%%%%
%%%%%%%%%%%%%%%%%%%
%%%%%%%%%%%%%%%%%%%
\section{Theory Summary}
The harmonics of glued straws, which have no motion at their ends, should theoretically follow
\begin{equation}
\label{main}
f^\text{glued}_n \approx   (\frac{\mathcal{K}_n\sqrt{m}}{2L}  +  \frac{\mathcal{C}_n}{L^2} )
\end{equation}
whereas the harmonics of pinned straws, which are allowed rotational but not transnational motion,  should follow
\begin{equation}
f^{\text{clamped}}_n =  \frac{n}{2L}\sqrt{\frac{T}{\mu}}  \sqrt{\frac{n^2\pi^2B}{TL^2} + 1}
\end{equation}
In particular, when the tension is zero, then
\begin{equation}
f^{\text{clamped}}_n =  \frac{\pi n^2}{2L^2} \sqrt{\frac{B}{\mu} } = \frac{\pi n}{2L^2} \mathcal{C}_n
\end{equation}
with
\begin{equation}
\mathcal{C}_n = n \sqrt{\frac{B}{\mu}}
\end{equation}
and 
\begin{equation}
\mathcal{K}_n = n\sqrt{\frac{g}{\mu}}
\end{equation}
Equation \ref{main} was confirmed in a past experiment, indicating we have at least partial understanding. However, the two parameters have theoretical values that are inconsistent with that same past experiment.

Which have theoretical values of 
\begin{equation}
\mathcal{K} =  \text{ 157.29  m s}^{-1} \text{kg}^{-1/2}
\end{equation}

\begin{equation}
\mathcal{C}_1
\end{equation}

 



%%%%%%%%%%%%%%%%%%%
%%%%%%%%%%%%%%%%%%%
%%%%%%%%%%%%%%%%%%%
\section{Direct Measurements}
As a cross check for the predicted frequency parameters $\mathcal{K}$ and $\mathcal{C}_n$, the values of $\mu$ and $B$ were measured for the PPG straws, which are the current straws being used for the panel construction. The Lamina straws that were used in the tests at Rice slightly differ from the PPG straws, which have a total of 800 \r{A} of aluminum and 100 \r{A} of gold. This difference in metallization should not measurably effect $\mu$ and is calculated to have only a few percent effect bending stiffness.  For $\mu$, the weight of a several meter long sample of straw-tubing was measured, while the latter was found by the vertical deflection, $\Delta y$ in the setup illustrated in Figure~\ref{fig:deflect} and the relation
\begin{equation}
B = \frac{7  \mu g L^4}{24 \Delta y}
\end{equation}
\begin{figure}[H]
	\includegraphics[width=0.37\textwidth]{pics/deflect}
	\caption{A segment of a straw}
	\label{fig:deflect}
\end{figure}
The outcome  of the density measurement was
\begin{equation}
	\mu_\text{mes} = \text{  (3.48 $\pm$ 0.06)e-4  kg m}^{-1}
\label{eq:mesmu}
\end{equation}
which yields
\begin{equation}
\boxed{
\mathcal{K}_\text{mes}  = \text{(167.8 $\pm$ 2.9) m s}^{-1} \text{kg}^{-1/2}
}
\end{equation}
Note that $\mu$ was measured at 10\% relative humidity, while the Rice tests were conducted at 50\%. Additionally, the straws at the rice tests were excessively handled, which raises $\mu$.  The effect of humidity and handling on $\mu$ has not been quantitatively studied however.


The measurement of the bending stiffness was found to be highly dependent on the straw's history and handling. To this extent, there was uncertainty in the measurements. It was found that
\begin{equation}
	B_\text{mes} = \text{  (3.6 $\pm$ 1.0 )e-3  N$\cdot$m}^2
\end{equation}
which is in agreement with both the predicted value in Equations~\ref{eq:b} and \ref{eq:totb}. 

A second approach was taken to measure $B$.  A straw was fixed at either end and held at zero tension. The straw was then plucked and its oscillations were recorded using a 240-fps slow motion camera.   In this scenario however Equation~\ref{eq:full} breaks down. Instead, the formula
\begin{equation}
	f \approx \frac{9 \pi}{8 L^2}\sqrt{\frac{B}{\mu}}
\end{equation}
can used to determine $B$ once $f$ is found by counting the number of oscillations in a given time interval.
 Note that the above equation was found numerically after from solving the original equations of motion with $T=0$, is a similar manner as was taken for large $T$.  The observed frequency gave a value of 
 \begin{equation}
 B_\text{count} = \text{  (2.7 $\pm$ 1.1 )e-3  N$\cdot$m}^2
 \end{equation}
which is on the smaller of the other estimates. The above value will be disregarded until a more systematic study is done.







\section{Second Order Frequency Effects}
A number of additional frequency effects were calculated and found to be insufficient to explain the discrepancy between predicted and measured parameter values. Such effects included the damping caused by air resistance, the reduction of $\mu$ due to straw stretching, second order terms in the frequency relationship, complex boundary conditions (treated in Section~\ref{sec:boun}), straw out of straightness, and shear forces. For the shear, it was found that
\begin{equation}
	\frac{B}{\kappa L^2 G A}\approx 0.0033 << 1
\end{equation}
for a one meter straw, which means the shear term in the equations of motion may be neglected. The out of straightness or s-shape of the straw can was similarly evaluated and also found to close to zero. 


 The effect of air resistance was found by considering an a damping term in the equation of motion, under the assumption of laminar flow.  The effect was found to be
\begin{equation}
	f_\text{air} = f\sqrt{1 - \frac{\gamma^2}{f^2}}
	\label{eq:air}
\end{equation}
where $\gamma$ is given by
\begin{equation}
\gamma = \frac{f A \rho_\text{air}}{M^2} (M  \sqrt{2} +  \frac{1}{2}  )
\end{equation}
and 
\begin{equation}
	M = \frac{\rho_\text{air} r^2 f \sqrt{2\pi} }{\eta_\text{air}}
\end{equation}
The scaling factor in Equation~\ref{eq:air} is plotted as a function of frequency in Figure~\ref{fig:drag} and found to be neglibablly close to 1. 
\begin{figure}[H]
	\includegraphics[width=0.45\textwidth]{pics/drag}
	\caption{The scaling factor of Equation~\ref{eq:air} is shown to be negligible for frequencies higher than a tenth of a Hertz}
	\label{fig:drag}
\end{figure}
This can be understood more simply since the decay time was found by~\cite{david} to be on the order of 1 s, which is much larger than $\frac{1}{f}$, therefore the system is under-damped and the frequency is altered neglibablly. 

In the following section, plots of the predicted frequency with the incorporation of higher order terms in $\xi$ and stretch corrections to $\mu$ are incorporated for the two extreme boundary condition scenarios, perfectly glued, and pinned (or hinged).  The stretch corrections to $\mu$ are not explicitly shown, but they have less then a percent effect on the frequency. The largest effect, as is clear from the plots, is the behavior at the boundary.



\section{Plots}
Here we include a number of plots of the Rice data and the predicted frequency  with the incorporation of higher order terms in $\xi$ and stretch corrections to $\mu$, for the two extreme boundary condition scenarios.  The measured frequencies, for the most part, lie somewhere in-between those predicted for perfectly glued and hinged boundary conditions. The curves shown use the measured linear density for the PPG straws in Equation~\ref{eq:mesmu} but the calculated bending stiffness for the lamina straws, as in Equation~\ref{eq:totb}.  The reason that the PPG linear density was used is because there was no direct measurement 


\label{sec:plots}
\begin{figure}[H]
	\includegraphics[width=0.47\textwidth]{pics/1}
	\caption{The Rice data overlaid with predictions for glued and pinned boundary conditions at various levels of approximation. ``Glued: 1st approx'' is the solution for perfectly glued boundary conditions, it includes up to second order terms in $\xi$ and refers to Equation~\ref{eq:full}. ``Glued 2nd approx'' includes up to first order terms in $\xi$ and refers to Equation~\ref{eq:end}. ``Pinned exact'' is the exact solution for pinned (or hinged) boundary conditions and refers to Equation~\ref{eq:pinsol}. ``Pinned approx'' is the first order expansion of the pinned solution and refers to Equation~\ref{eq:pinapprox}. The curves shown use the the calculated bending stiffness for the lamina straws, as in Equation~\ref{eq:totb}, and, because there are no direct measurements for the linear mass density of the Lamina straws, the measured density of the PPG straws, as in Equation~\ref{eq:mesmu} was used.
	}
	\label{fig:1}
\end{figure}
\begin{figure}[H]
	\includegraphics[width=0.47\textwidth]{pics/2}
	\caption{The Rice data overlaid with predictions for glued and pinned boundary conditions at various levels of approximation. See Figure~\ref{fig:1} for a more complete description. 
	}
	\label{fig:2}
\end{figure}
\begin{figure}[H]
	\includegraphics[width=0.47\textwidth]{pics/3}
	\caption{The Rice data overlaid with predictions for glued and pinned boundary conditions at various levels of approximation. See Figure~\ref{fig:1} for a more complete description. 
	}
	\label{fig:3}
\end{figure}
\begin{figure}[H]
	\includegraphics[width=0.47\textwidth]{pics/4}
	\caption{The Rice data overlaid with predictions for glued and pinned boundary conditions at various levels of approximation. See Figure~\ref{fig:1} for a more complete description. 
	}
	\label{fig:4}
\end{figure}
\begin{figure}[H]
	\includegraphics[width=0.47\textwidth]{pics/5}
	\caption{The Rice data overlaid with predictions for glued and pinned boundary conditions at various levels of approximation. See Figure~\ref{fig:1} for a more complete description. 
	}
	\label{fig:5}
\end{figure}
\begin{figure}[H]
	\includegraphics[width=0.47\textwidth]{pics/6}
	\caption{The Rice data overlaid with predictions for glued and pinned boundary conditions at various levels of approximation. See Figure~\ref{fig:1} for a more complete description. 
	}
	\label{fig:6}
\end{figure}
\begin{figure}[H]
	\includegraphics[width=0.47\textwidth]{pics/7}
	\caption{The Rice data overlaid with predictions for glued and pinned boundary conditions at various levels of approximation. See Figure~\ref{fig:1} for a more complete description. 
	}
	\label{fig:7}
\end{figure}



\section{Conclusion}
By the inclusion of restoring forces that arise both from tension applied and bending stiffness intrinsic to the Lamina straws, we obtained an equation of motion that, given the appropriate boundary conditions, yields natural frequencies with a functional form identical to what was found experimentally. However, discrepancies between the predicted and experimental values of two parameters were found.  Several explanations for the discrepancies have been ruled out, including many additional restoring forces in the equation of motion, by virtue of their calculated smallness. While the reason for the discrepancy is at present unclear, the plots of Section~\ref{sec:plots} indicate that the Lamina straws used by\cite{david} may have had a slightly larger linear mass density than what was assumed, which could arise from the fact that they were manufactured differently than the well characterized PPG straws, and through excessive handling and humidity at the time of testing. Additionally, the plots may indicate that the straws had a non-trivial behavior at the boundaries. Parameter values aside, the functional form of the natural frequencies for the mu2e straw-tubes are now understood.




\section{More Direct Measurements}
In the limit as tension goes to zero, equation~\ref{eq:clamped2} becomes
\begin{equation}
f_1 =  \frac{\pi}{2L^2}\sqrt{\frac{B}{\mu}} = \frac{\pi\mathcal{C}}{2L^2}
\end{equation}

\begin{equation}
\lim_{T\to0} f^{\text{pinned}}_1 = \frac{\pi\mathcal{C}}{2L^2}
\end{equation}


\begin{figure}[H]
	\includegraphics[width=0.47\textwidth]{pics/quick}
	\caption{stuff
	}
	\label{fig:quick}
\end{figure}


\begin{equation}
\mathcal{C} \equiv \sqrt{\frac{B}{\mu}}
\end{equation}


\begin{equation}
	\mathcal{K} \equiv \sqrt{\frac{g}{\mu}} = \text{ 157.29  m s}^{-1} \text{kg}^{-1/2}
\end{equation}  



\begin{equation}
	f _1= \frac{\mathcal{K}}{2L}\sqrt{m} + \frac{\mathcal{C}}{L^2}
\end{equation}

\begin{equation}
	f _1= \frac{\mathcal{K}}{2L_\text{eff}}\sqrt{m_\text{eff}} + \frac{\mathcal{C}}{L_\text{eff}^2}
\end{equation}



\begin{equation}
	f _1= \frac{\sqrt{m}\mathcal{K}}{2L}\sqrt{\frac{\pi^2\mathcal{C}^2}{L^2} + 1}
\end{equation}


\begin{equation}
	B = \frac{5 \mu g L^4}{384 \Delta y}
\end{equation}

The length of the straw under tension is 
\begin{equation}
	L_{\text{eff}} = L(1+TC_s)
\end{equation}
where the stretch coefficient $C_s = 8.9e-3 \frac{\Delta L}{L \text{Kg}}$ . So we have 
\begin{equation}
	\mu_{\text{eff}} = \frac{\mu}{(1+TC_s)}
\end{equation}
and thus,
\begin{equation}
	%\mu_{\text{eff}} = \frac{\mu}{(1+TC_s)}
	\mathcal{C}_\text{eff} = \sqrt{\frac{B}{\mu_{\text{eff}}}} = \sqrt{\frac{B(1+TC_s)}{\mu}} = \mathcal{C}\sqrt{(1+TC_s)}
\end{equation}
and
\begin{equation}
	%\mu_{\text{eff}} = \frac{\mu}{(1+TC_s)}
	\mathcal{K}_\text{eff} = \sqrt{\frac{g}{\mu_{\text{eff}}}} = \sqrt{\frac{g(1+TC_s)}{\mu}} = \mathcal{K}\sqrt{(1+TC_s)}
\end{equation}

\begin{equation}
	\mathcal{K}_\text{eff} = \mathcal{K}\sqrt{(1+TC_s)}
\end{equation}

However, $\mathcal{C}$ is proportional to $(r_\text{out}^4 - r_\text{inner}^4)$.  The corrected cross sectional area is 
\begin{equation}
A_\text{eff} = \frac{A}{(1+TC_s)} = \pi \frac{(R^2 - r^2)}{(1+TC_s)} 
\end{equation}
\begin{equation}
B_\text{eff} \approx \frac{B}{1+TC_s}
\end{equation}
\begin{equation}
\mathcal{C}_\text{eff} = \mathcal{C}
\end{equation}

\begin{equation}
m_\text{eff} = 0.983(m + \frac{15}{L 1000}) 
\end{equation}



\begin{equation}
(R^2 - r^2)(R^2 + r^2) = R^4 - r^4
\end{equation}
But the second factor, $(R^2 + r^2)$, is not really changed, so you end up with a factor of $\frac{1}{1 + TC_s}$ to correct the bending stiffness so $\mathcal{C}$ is unchanged!!!
\begin{equation}
\mathcal{K} = \frac{2}{\sqrt{m}} (Lf_1 - \frac{\mathcal{C}}{L})
\end{equation}

\begin{equation}
\mathcal{C} =  (L^2 f_1 - \frac{L\mathcal{K}\sqrt{m}}{2} )
\end{equation}


\begin{equation}
I = \frac{\pi}{4}(R^4 - r^4)
\end{equation}

So the screw actually moved the straws in the direction that lessens the tension.  For a quarter turn, which is about what Arlette did, it moved about 20 mil, which is 0.000508 meters. Thus since Cs = 8.9e-3, we have a change in tension for a 1 meter straw of about 57 g!!! thats huge.

straw strech:

$E = \frac{1}{C_s A}$ \\
$B = E I$ \\
$C = \sqrt{\frac{B}{\mu}}$

%----------------------------------------------------------------------------------------
%	REFERENCE LIST
%----------------------------------------------------------------------------------------

\begin{thebibliography}{99} % Bibliography - this is intentionally simple in this template

\bibitem{tdr}
Mu2e TDR 

\bibitem{fletcher}
Fletcher; The Physics of Musical Instruments

\bibitem{david}
David Rivera; Rice University Senior Thesis

\bibitem{dick}
JC Bryner;  Richard Feynman on piano tuning



%\bibitem{zhao}
%T. Zhao et al, \ 1993, NIM A 340 485-490

\end{thebibliography}

%----------------------------------------------------------------------------------------

\end{multicols}

\end{document}




